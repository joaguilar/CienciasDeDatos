\documentclass[letterpaper, 11pt]{article}
\usepackage{comment} % habilita el uso de comentarios en varias lineas (\ifx \fi) 
\usepackage{lipsum} %Este paquete genera texto del tipo  Lorem Ipsum. 
\usepackage{fullpage} % cambia el margen
\usepackage[utf8]{inputenc}
\usepackage[spanish]{babel} 


\begin{document}
\linespread{1.5}
\noindent
\large\textbf{Tarea 1} \\
\textbf{Resumen del libro "How to Lie with Statistics" de Darrell Huff} \\
\normalsize Matemática para la Ciencia de Datos \\
Alumno: Jose Aguilar\hfill 2/Abril/2019 \\


\section*{Capítulo 1: La muestra con el sesgo incorporado}
\selectlanguage{spanish}

En el reportaje de Yale está la primera gran enseñanza del libro. En este se menciona el ingreso promedio de \textdollar{25.000} de los egresados de Yale, y utilizando este ejemplo muestra los errores en el muestreo, los cuales se dan porque es más fácil encontrar a los egresados a los que les ha ido bien económicamente, con lo que se sesga esta cifra hacia esos individuos de forma importante. Aquellos egresados a los que no les ha ido tan bien, o no responden o no son localizables por esta misma razon. Además, se utiliza un ejemplo de unas encuestas de las revistas que leen las personas para mostrar que aún cuando son localizables, no siempre dicen la verdad.

\textit{El resultado de un estudio con un muestreo es tan bueno como la muestra en qué se basa.}

La muestra puede tener sesgos incorporados, tal como lo muestran los ejemplos de tasas de sobrevivencia al cancer y la afirmación del psiquiatra de que \textit{todo el mundo es neurótico}. Además la muestra puede tener sesgos invisibles, por lo que el libro recomienda ser escépticos, aún cuando se ven y se consideran sesgos obvios. Como le ejemplo que dan Tere elección del 32, y que se puede pensar ni algo parecido en las últimas elecciones en Costa Rica. La muestra basica es llamada \textit{aleatoria}, y es seleccionada al azar del universo en estadística significa el todo del cual la muestra es una parte.

El muestreo aleatorio es el único que puede ser analizado estadísticamente con confianza, pero tiene limitaciones de costo. Por esto existe el muestreo aleatorio estratificado y que es una opción más económica, por lo que es lo más usado.  Para utilizar este tipo de muestreo se divide el universo en grupos de acuerdo a su prevalencia, de tal forma que tengan la misma proporción. El problema es que esa proporción puede que no sea la correcta. Y obtener una muestra aleatoria dentro de cada estrato es igual de caro. Es una guerra continua contra los orígenes de los ascos. Es importante tener en cuenta que la guerra nunca se gana por lo que hay que ser muy escéptico de los titulares de las noticias.
 
 Algo interesante es que se hace referencia a un estudio del Dr. Alfred Kinsey donde se notan tres niveles de muestreo en los cuales se pueden introducir un sesgo el muestreo de polo población el muestreo de las preguntas posibles y el maestro de las respuestas que nos alejan las actitudes y experiencias de los entrevistados con respecto a las preguntas.
 
 También es necesario ver como los entrevistadores afectan las respuestas. Como el ejemplo dado del N o R FC y como saber entrevistadores blancos o negros causaba que los entrevistados dieron respuestas diferentes lo cual que se considera consenso introducido por factores desconocidos en las encuestas hay que considerar la tendencia a dar una respuesta que sea satisfactoria para el encuestador.
 
 Esto hace pensar que las respuestas en esos casos están sesgadas y que no vale nada por eso hay que pensar en todos los casos donde las conclusiones de una encuesta tiene algo sesgo que no es conocido esto es muy obvio en las encuestas políticas y recientemente tenemos ejemplos claros como en las elecciones de Estados Unidos en el 2016 el referendo en Gran Bretaña para salir de la unión europea inclusive las elecciones de Costa Rica en el 2018.
 
 Este sesgo mencionado normalmente eso hace la persona con más dinero mayor educación mejor información más alerta con mejor apariencia con un comportamiento más convencional y más hábitos que la persona promedio de la población que representa.Esto es consistente con el sentimiento en contra de las encuestas de opinión que se encuentra en los círculos de izquierda donde normalmente se dice que las encuestas están modificadas. Detrás de esta visión está el hecho de qué no haz encuestas normalmenteEstán sesgados a favor de los comportamientos conservadores esto hace que no necesariamente se modifique los resultados de una encuesta para favorecer un grupo simplemente la tendencia de muestreo puede estar sentada en una dirección consistentemente.
 
\section*{Capítulo 2: el promedio bien escogido}

El promedio es una de las herramientas más utilizadas para mentir con estadísticas. Existe tanto la media que es el promedio aritmético,Como también existe la mediana qué parte del grupo en dos de tal manera que la mitad de los números usados es menor que la mediana y la otra mitad mayor que la mediana y finalmente está la moda que es el número más como dentro de los datos de entrada. Por esto cuando alguien dice promedio Hay que estar bien claro de cuál de estos se está refiriendo.  Esos promedios normalmente se encuentran cerca uno del otro sobre todo cuando se trabaja con características humanas dado que los humanos caen en lo que se conoce como la distribución normal por lo cual los tres tipos de promedios siempre son similares. Si embargo para otro tipo de datos, los ingresos esto puede ser muy diferente por ejemplo si existen unos pocos individuos con ingresos realmente alto pueden ser sacar significativamente la media sin embargo la mediana reflejaría mejor como es el ingreso aproximado del grupo. Esto depende de como estan distribuidos los datos - en el caso de los ingresos no es una distribución normal, sino una distribución tirada a un lado.

El ejemplo que se da en el libro sobre los salarios en una empresa es muy revelador. Esto aplica a un el día de hoy donde irnos ejecutivos de los corporaciones tienen salarios que son muchas veces el salario de los empleados matas. Si se saca la media de los salarios no se va a reflejar realmente cuánto es que ganan Las personas en esa corporación. La moda probablemente va a estar más cerca de los salarios de los operadores de bajo nivel de la empresa por lo que la mediana probablemente sea mucho menor que la media.

También es muy interesante ver como se manipulan estos promedios para aparentar cosas que no necesariamente son Sharks ciertas. Como por ejemplo el ejemplo que dan en los lectores de la revista también que dependiendo de cual encuestas ahí revisa el ingreso promedio es diferente por lo que cabe dudar sobre su veracidad.

En resumen, TODO

\section*{Capítulo 3: Las pequeñas figuras que no están ahi}

El capitulo abre con otro ejemplo muy interesante sobre como manipular con estadisticas: \textbf{Los usuarios reportan 23\% menos caries con la pasta de dientes \textit{Doakes}}. Para llegar a este número, utilizaron grupos muy pequeños y probaron con diferentes grupos hasta que encontraron con un número que les satisfaciera. Esto muestra una de las principales enseñanzas del capítulo la cual es que en un grupo con suficiente tamaño cualquier diferencia en los resultados producidos por el azar va a ser muy pequeña. Sin embargo resultados que al final de cuentas serían aleatorio y no indicarían nada en particularPueden ser producidos por el azar dado un pequeño grupo de casos cualquier resultado es posible. Esto lo ilustra muy bien con el ejemplo de las monedas de tirar una moneda varias veces y en algunos escenarios caerá la mayoría importante de las veces en una cara y en otros escenarios en la otra.Por lo que uno podría tomar cualquiera de sus escenarios que más le convenga y utilizar eso como resultados del estudio facilitado si uno no muestra la metodología que se utilizó.

\textit{Solamente cuando hay un suficiente número de pruebas involucradas es que la ley de los promedios produce una descripción o predicción adecuada sobre las observaciones.}

Éste número de suficiente número de pruebas es algo a lo que es difícil llegar esto lo ilustra muy bien el libro con el ejemplo de los casos de polio donde la incidencia de pollo paralítico está baja que se necesita un número real mente grande para poder observar alguna diferencia. Para esto existeLa prueba de significancia que simplemente reporta qué tan probable es que una prueba representa un resultado Real en vez de un resultado que se produjo al azar. Éste grado de significancia normalmente se expresado como una probabilidad. Si la fuente información contiene este dato es más probable que sea un estudio serio y permite darse una mejor idea de los datos que se observaron al hacer el estudio. Se citan los ejemplos del tamaño de la familia, 

La importancia de conocer el grado de significancia, que normalmente se representa como un margen alrededor del resultado, es explicada bastante bien a travez del ejemplo de los \textit{Gessel Norms} que son observacions sobre las edades a las que los niños aprenden ciertos comportamientos. Estos se describen como \textit{normales}, pero mas normales en términos estadísticos que su uso como algo \textit{deseable}, como en este caso que causó angustia a padres de familia porque los niños no se sentaban a las "x" cantidad de meses. Asi también se da el ejemplo de los estudios de sexualidad del Dr. Kinsey.

La ausencia del grado de significancia normalmente no es algo que moleste a los periodistas - con solo reportar los numeros principales de un estudio, los periodistas pueden crear un titular espectacular que el dia de hoy puede generar muchos clicks, sin tener que molestarse en entender realmente que es lo que dice un estudio. 

Finalmente el capítulo habla sobre como este tipo de deshonestidad se ve reflejada en gráficos como el del crecimiento del negocio de la agencia de publicidad - sin nigun tipo de cifra ni contexto. El autor visita de nuevo este temad e los gráficos en el capítulo 5.


\section*{Capítulo 4: Mucho ruido acerca de prácticamente nada}

Se utiliza un ejemplo de coeficiente intelectual para reflejar dos conceptos sumamente importantes en las estadísticas: el error probable y el error estándar. Éstos representan qué tan precisos pueden ser las muestras que representan el universo de los datos. El error probable habla sobre en promedio cuánto es el error en las mediciones - como en el ejemplo se menciona que al medir 100 yardas, se mide la mitad de las veces 97 yardas, y la otra mitad mas de 3 yardas. El error estándar es similar pero considera dos terceras partes de los casos en vez de la mitad.
El error estándar normalmente se denota como un porcentaje, un \textit{mas/menos \textbf{+/-}} con respecto a la medición reportada, como por ejemplo 100+/-3 en el caso de la medición de las 100 yardas.
Ignorar estos errores que son implícitos en cualquier estudio con muestreo puede causar errores de interpretación importante, como el que se menciona de los editores de revistas que se basan en las encuestas de sus lectores y que los pueden llevar a caer en conclusiones erróneas. El error probable que se encuentra en estas encuestas puede ser tan grande como para que no significan nada trascendental.
Otro ejemplo muy interesante que se da en el libro es el de los cigarrillos cuando hay diferencias muy pequeñas en cuanto a los químicos tóxicos en el humo del cigarro de acuerdo a un estudio realizado. Una de las marcas analizadas encontró que pese a que la diferencia En estos químicos era mínima con respecto a las demás, y el libro inclusive da entender un poco que pudo haber estado dentro de este error estándar, esta marca de cigarrillos aprovecho esos números, sin importarles Lo pequeño que fueran Y lo utilizo para una campaña comercial que termino con una orden de desistir de usar ese tipo de publicidad engañosa.

\section*{Capítulo 5: El gráfico vistoso}

Debido a la resistencia que tienen las personas a trabajar con números y a las matemáticas en general, se recurre a gráficos para representar la información. El tipo de gráfico más común es el gráfico de líneas que muestran tendencias, lo que es algo en lo que cualquiera tiene interés para realizar predicciones. Estos gráficos, sin embargo, son muy Dados a ser manipulados con la intención de ganar un argumento, impresionar lector, o venderle algo. Por esto el autor menciona algunas técnicas que utilizan para este efecto, como por ejemplo el eliminar los márgenes, O en cogerlos, con el fin de dar la impresión que por ejemplo una tendencia es mucho más alta que los valores realmente representa. Aparte de eliminar los márgenes otro truco comúnmente usado es el de comprimir o estirar los márgenes, haciendo que los \textit{pasos} entre número en los ejes del gráfico sea mas pequeño o  de tal manera que las tendencias se vean más exageradas y causen un efecto mas impresionante en el lector del gráfico.
 Otro ejemplio que se menciona es el de ajustar los ejes, cortandoles la parte inferior, de tal forma que se puede accentuar mas o menos una diferencia, como el crecimiento economico, crecimiento en ventas, diferencias entre dos valores, etc - lo cual es similar a lo que se hace 

\section*{Capítulo 6: El dibujo de una sola dimensión}

 Los cuadros tipos pictogramas son otro aspecto que el libro menciona como una herramienta para ajustar la narrativa a lo que se quiere que el público interprete. El ejemplo dado donde se utiliza el crecimiento exponencial del volumen de un objeto (bolsas de dinero o el gráfico sobre la expectativa de vida de los estadounidenses, que al final de cuentas son 8 veces mas grandes en la imaginación de las personas en vez del doble que dicen los numeros, o el horno para fundir metales que es 3 veces mas grande en vez de 1,5 veces). Es curioso que se mencione de forma divertida el escenario de las vacas - que al triplicar el tamaño de las vacas para la representación en la revista, se da la impresion que las vacas de 1936 son \textbf{mucho mas pequeñas} que las vacas de 1936 para el lector distraído - y desafortunadamente da lo mismo con el rinoceronte, pero a la inversa.

 Al final de cuentas siempre hay que desconfiar en los casos que, tal y como dice el libro, de estos pictogramas, y analizar mas los números que su representación pictografica.
 
 Cabe destacar que en el día de hoy existen los famosos \textbf{infogramas}, que cumplen una función similar - detrás de muchos de estos infogramas hay una agenda que quiere mover la opinión de quién lo mira. 
 
 \section*{Capítulo 7: La figura semi-adjuntada}

\section*{Capítulo 8: \textit{Post Hoc} viene de nuevo}

Este capítulo expone una de las formas más comunes de cometer errores a la hora de extraer conclusiones sobre 

En inglés se dice normalmente \textit{Correlation is not Causation}, lo cual refleja claramenete lo que expone el capítulo. 

\section*{Capítulo 9: Como \textit{Estaticular}}

\section*{Capítulo 10: Como contestarle a una estadística}

\section*{Opinión y Conclusiones}

\bibliographystyle{unsrt}.
\bibliography{Jose_Aguilar_Tarea1.bib}



\end{document}